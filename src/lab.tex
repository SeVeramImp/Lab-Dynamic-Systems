\documentclass[12pt]{article}

\usepackage[utf8]{inputenc}
\usepackage[T2A]{fontenc}
\usepackage[russian]{babel}
\usepackage{amsmath, amssymb, amsthm}
\usepackage{graphicx}
\usepackage{geometry}
\usepackage{indentfirst}
\geometry{margin=2.5cm}

\begin{document}

\begin{titlepage}
    \centering
    \includegraphics[width=0.6\textwidth]{images/logistic.png}

    \vfill

    {\Large \textbf{Лабораторная работа}: \\ Исследование точечных отображений\\}
    \vspace{2.5cm}
    {\raggedleft \large Выполнил: Дорогостайский \\ Илья Ярославович, 
    \\ группа J3114\\}

    \vfill

    {\today \\ Санкт-Петербург}
\end{titlepage}

\section*{Введение}
\vspace{0.8em}

В данной лабораторной работе будем исследовать поведение дискретных динамических систем, задаваемых точечным отображением
\[\vec{x}_{n+1} = \vec{f}(\vec{x}_n),\]
где $\vec{x}_n$ — состояние системы в момент $n$, а $\vec{f}$ —
отображение, определяющее переход к следующему состоянию, на примере логистического отображения и его модификаций.
\vspace{0.8em}

Будем использовать понятия предела, монотонности и ограниченности, изученные в рамках курса математического анализа, в данной работе для исследования рассматриваемых в качестве примера динамических систем последовательностей.
\vspace{0.8em}

\textbf{Смысл работы} состоит в закреплении полученных знаниях на практике. Наша задача - научиться исследовать поведение посследовательностей и изучить свойства последовательностей, задаваемых рекуррентно.

\vspace{0.8em}
\textbf{Ход работы:}
\begin{itemize}
    \item Проанализируем динамику логистического отображения
    при различных значениях параметров и начального условия.
    
    \item Исследуем влияние параметра $r$ на поведение точечных отображений и изменение их качественных свойств.
    
    \item Ознакомимся с понятием неподвижной точки, определим условия её существования и количество.
    
    \item Изучим последовательности, порождаемые отображением, на предмет монотонности и наличия предела.
    
    \item Проанализируем существование циклов порядка $m$ и их свойства для рассматриваемых отображений.
    
    \item Исследуем поведение циклов разных порядков с использованием построения лестниц Ламерея.
    
    \item Подтвердим полученные аналитические результаты с помощью графической визуализации.
\end{itemize}
\clearpage


\section*{Easy}
Прежде чем приступить к практике, обозначим вводимые в рамках работы определения и поясним их.

\textbf{Определение:}
\textit{Логистическим отображением} называется функция вида
\[x_{n+1} = r x_n (1 - x_n),\]
где $r$ — параметр скорости прироста, $x_n$ - состояние системы в некий момент $n$, а $x_{n+1}$ - её же состояние в следующий момент. Также известны свойства логистического отображения, такие как:
\[r \in [0,4],\]
\[\forall n \in \mathbb{N}_0 \quad x_n \in [0,1].\]

\small{\textbf{\textit{NB}:} Логистическое отображение широко используется при описании динамических систем, например при исследовании скорости изменения численности популяций. В таком случае параметр $r$ характеризует скорость роста популяции от момента $n$ до момента $n+1$ (фактор "рождаемости минус смертности")}

\subsection*{Задача 1}
\textbf{Доказать утверждение:}
\[\forall n \in \mathbb{N}, \; \forall r \in (0,1]:
\quad
0 < x_0 < 1 \; \Longrightarrow \; 0 < x_n < 1.\]

\begin{proof}
(по методу математической индукции) \\

\textbf{\small{1) База индукции:}}

По определению логистического отображения $x_n = r x_{n-1} (1 - x_{n-1}).$ 

Тогда при $n=1$ имеем:
\[x_1 = r x_0(1-x_0).\]

Так как $0<x_0<1$, имеем: \[-1<-x_0<0 \iff 0<1-x_0<1\]

Тогда $0<x_0(1-x_0)<1.$ При $r \in (0,1]$ получаем:\[0 < r x_0(1-x_0) < 1.\]

Итак,\[0<x_1<1.\]

\textbf{\small{2) Индукционное предположение:}}

Пусть для некоторого $k = n$ верно:
\[0<x_k<1.\]

Тогда докажем, что и $0<x_{k+1}<1$ через определение логистического отображения:
\[x_{k+1} = r x_k (1 - x_k).\]

Так как $0 < x_k < 1$, то $0 < 1 - x_k < 1.$ Следовательно, $0 < x_k(1 - x_k) < 1.$

При $r \in (0,1]$, получаем
\[0 < r x_k(1 - x_k) < 1.\]

Следовательно,
\[0 < x_{k+1} < 1.\]

\textbf{\small{3) Интерпретация:}}

Значит при $x_0 \in (0,1)$ верно:
\[\forall n \in \mathbb{N}, \; \forall r \in (0,1]:
\quad 0 < x_n < 1.\]

Действительно, если про начальное состояние $x_0$ нам известно, что $x_0 \in (0,1)$, тогда, исходя из формулы логистического отображения и ограничений, наложенных на $r$, после $n$ итераций мы получим $x_n$, также не выходящее за пределы $(0,1)$. Т.е. справедливо будет сказать:
\[\forall n \in \mathbb{N}, \; \forall r \in (0,1]:
\quad
0 < x_0 < 1 \; \Longrightarrow \; 0 < x_n < 1.\]
\end{proof}

\subsection*{Задача 2}
\textbf{1. Как параметр $r$ влияет на поведение функции зависимости $x_{n+1}$ от $x_n$?}

Заметим, что график зависимости $x_{n+1}$ от $x_n$ — это график функции
\[y=f(x)=rx(1-x).\]

Рассмотрим этот график, чтобы изучить влияние параметра $r$ на поведение функции:

\begin{itemize}
    \item $f(0)=0$ и $f(1)=0$, т.е. график проходит через точки $(0,0)$ и $(1,0)$ при любом $r$.
    \item $f(x)=rx(1-x)=-rx^2+rx$ - квадратичная функция, график является параболой с ветвями, направленными вниз.
    \item Вершина параболы достигается при $x=\frac12$:
    \[f\!\left(\frac12\right)=r\cdot\frac12\left(1-\frac12\right)=\frac r4.\]
    Следовательно, параметр $r$ масштабирует график по вертикали: чем больше $r$, тем выше вершина и тем больше значения $x_n$ при фиксированном $x_{n-1}\in(0,1)$.

    \item Так как $\max f(x)=\frac r4$, то при $r \in [0, 4]$ имеем $f(x) \leq 1$ на $[0,1]$, что согласуется с оценкой $x_n \in [0,1].$ При $r>4$ член $x_n$ может оказаться вне этого диапазона.
    \item \textbf{Вывод:} Параметр $r$ не меняет значения $f$ на $x=0, \quad x=1$, но изменяет высоту и крутизну графика: при росте $r$ парабола растягивается вверх, а максимум $\max f(x)=\frac r4$ растёт линейно.
\end{itemize}

\textbf{2. Постройте эту функцию для нескольких различных значений r.}
\begin{figure}[h!]
\centering
\includegraphics[width=1.0\textwidth]{images/logistic_map.png}
\caption{Графики функции $x_{n+1} = r x_{n}(1-x_{n})$ при различных $r$.}
\end{figure}

\subsection*{Задача 3}
\textbf{1. Постройте графики зависимости $x_{n+1}$ от $x_n$ для различных значений $r$.}

По условию варианта ($N=0$) будем рассматривать точечное отображение
\[x_{n+1} = r x_n(1-x_n)(2+x_n), \qquad r \in \left[0;\; \frac{27}{2(7\sqrt{7}-10)}\right].\]

Рассмотрим график этой функции:
\begin{itemize}
    \item $g(0)=0$ и $ g(1)=0.$ Следовательно, график проходит через точки $(0,0)$ и $(1,0)$. Также существует нуль при $x=-2$, однако он не принадлежит отрезку $[0,1]$.
    \item $g(x)=r(2x-x^2-x^3)$ - кубическая функция, графиком является кубическая парабола. 
    \item При $x\in[0,1]$ выполняется: $x \ge 0$; $1-x \ge 0$; $2+x > 0,$ следовательно, \[g(x) = r x(1-x)(2+x) \ge 0.\]
    \item Чтобы найти максимум функции, рассмотрим её производную: \[g'(x)=r(2-2x-3x^2).\]

Из уравнения $g'(x)=0$ получаем: \[2-2x-3x^2=0 \quad \Rightarrow \quad x=\frac{\sqrt7-1}{3}.\]

Эта точка принадлежит $(0,1)$ (т.к. $\frac{\sqrt7-1}{3} \approx 0.55)$ Данная точка экстремума единственна на рассматриваемом промежутке, и проходя через неё $g'(x)$ меняет свой знак с $+$ на $-$, следовательно, $x = \frac{\sqrt7-1}{3}$ - точка максимума функции на $(0,1)$.

    \item Подставим это значение, чтобы оценить $r$:
\[\max g(x)=r(2\frac{\sqrt7-1}{3}-\frac{(\sqrt7-1)^2}{9}-\frac{(\sqrt7-1)^3}{9})=\] 
\[= r(2\frac{\sqrt7-1}{3} - \frac{8-2\sqrt7}{9} - \frac{10\sqrt7-22}{27}) = r\frac{18(\sqrt7-1)-3(8-2\sqrt7)-(10\sqrt7-22)}{27} =\] \[= r\frac{2(7\sqrt7-10)}{27}.\]

Из $\max g(x)\le 1$, следует \[r \le \frac{27}{2(7\sqrt7-10)}.\]
\end{itemize}

\begin{figure}[h!]
\centering
\includegraphics[width=1.0\textwidth]{images/N0_map.png}
\caption{Графики функции $x_{n+1} = r x_{n}(1-x_{n})(2+x_{n})$ при различных $r$.}
\end{figure}
\vspace{0.8em}

\textbf{2.  Сделайте вывод о сходстве или различии поведения логистического отображения и точечного отображения из вашего варианта. Чем могут быть вызваны сходства/различия?} \\

Исходя из рассмотренных свойств функции $g(x) = rx(1-x)(2+x)$, нетрудно выявить её сходства и различия сравнительно с функцией логистического отображения $f(x) = rx(1-x)$:

\vspace*{0.5cm}
\textbf{Сходства.}
\begin{itemize}
    \item В обоих случаях имеется точечное отображение вида $x_{n+1} = F(x_{n})$, зависящее от параметра $r$, причём увеличение $r$ приводит к увеличению значений $F(x)$ при фиксированном $x\in(0,1)$ ($r$ вертикально масштабирует график).
    \item Оба отображения имеют нули на $[0,1]$ в точках $x=0$ и $x=1$: \[f(0)=f(1)=0, \qquad g(0)=g(1)=0,\] поэтому их графики проходят через точки $(0,0)$ и $(1,0)$, вне зависимости от $r$.
\end{itemize}

\textbf{Различия.}
\begin{itemize}
    \item Логистическое отображение задаёт квадратичную функцию, а отображение $g(x)=r x(1-x)(2+x)$ - кубическую. Поэтому форма графиков будет отличаться: хотя на [0,1] внешне графики и схожи, кривая $g(x)$ не является параболой $f(x)$ и имеет иное распределение значений по $x$ (даже при фиксированном $r$).
\end{itemize}

Очевидно, что сходства обусловлены общей структурой: оба отображения задаются итеративно и зависят от параметра $r$ как от масштабирующего множителя. \\

В свою очередь, различия вызваны разной старшей степенью отображений, что меняет форму графика $x_{n+1}$ от $x_{n}$ и, как следствие, характер итераций.
\clearpage


\section*{Normal}
Прежде чем приступать к дальнейшим заданиям, рассмотрим вводимое в рамках работы определение неподвижной точки.

\textbf{Определение.}
Точка $x^*$ называется \textit{неподвижной точкой} отображения $f(x)$, если \[x^* = f(x^*).\]
Иными словами, при применении отображения неподвижная точка переходит сама в себя.

\subsection*{Задача 4}
\textbf{1. Найти все неподвижные точки логистического отображения.}

Рассматривая $x_n = x^*$ в логистическом отображении $x_{n+1} = r x_n (1 - x_n)$, справедливо будет заметить, что:
\[x_{n} = r x_n (1 - x_n).\]

Найдём неподвижные точки, удовлетворяющие этому равенству, решив его:
\[x = r x (1 - x) \Longleftrightarrow r x - r x^2 - x = 0 \Longleftrightarrow\] 
\[ \Longleftrightarrow - r x^2 + (r - 1)x = 0 \Longleftrightarrow\ x \big( - r x + r - 1 \big) = 0\]

Отсюда получаем
\[x = 0,\]
\[- r x + r - 1 = 0.\]

Из второго уравнения:
\[r x = r - 1 \Longleftrightarrow\ x = \frac{r - 1}{r}\]

Итак, мы нашли неподвижные точки:
\[x_1 = 0, \qquad x_2 = \frac{r - 1}{r}.\]

\textbf{2. При каких $r$ отображение имеет одну неподвижную точку? Несколько?}

При $r = 0$, отображение $f(x) = r x (1 - x) = 0$, и уравнение $x = f(x)$ даёт единственное решение: \[x = 0.\]

При $r = 1$, согласно найдённой неподвижной точке, имеем: \[x_2 = \frac{1 - 1}{1} = 0,\]
т.e. обе формулы дают одну и ту же точку. Следовательно, при $r = 1$ неподвижная точка одна. Итак:
\[\begin{cases}
r = 0 \cup r = 1 & \text{одна неподвижная точка},\\
r \in (0,1) & \text{две различные неподвижные точки}.
\end{cases}\]
\vspace{0.8 cm}

\textbf{3. Какое максимальное количество неподвижных точек может иметь логистическое отображение? Почему?}

Уравнение неподвижных точек приводится к квадратному уравнению относительно $x$:
\[x = r x (1 - x) \Longleftrightarrow - r x^2 + (r - 1)x = 0.\]

Квадратное уравнение может иметь не более двух действительных корней. Следовательно, максимальное количество неподвижных точек логистического отображения равно двум.
\subsection*{Задача 5}
\textbf{}

\end{document}
