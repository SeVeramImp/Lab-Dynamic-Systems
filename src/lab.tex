\documentclass[12pt]{article}

\usepackage[utf8]{inputenc}
\usepackage[T2A]{fontenc}
\usepackage{listings}
\usepackage{xcolor}
\lstset{
    language=Python,
    basicstyle=\ttfamily\small,
    keywordstyle=\color{blue},
    stringstyle=\color{teal},
    commentstyle=\color{gray},
    numbers=left,
    numberstyle=\tiny,
    stepnumber=1,
    numbersep=8pt,
    frame=single,
    breaklines=true,
    tabsize=4,
    showstringspaces=false
}
\usepackage[russian]{babel}
\usepackage{amsmath, amssymb, amsthm}
\usepackage{graphicx}
\usepackage{geometry}
\usepackage{indentfirst}
\geometry{margin=2.5cm}

\begin{document}

\begin{titlepage}
    \centering
    \includegraphics[width=0.6\textwidth]{images/logistic.png}

    \vfill

    {\Large \textbf{Лабораторная работа}: \\ Исследование точечных отображений\\}
    \vspace{2.5cm}
    {\raggedleft \large Выполнил: Дорогостайский \\ Илья Ярославович, 
    \\ группа J3114\\}

    \vfill

    {\today \\ Санкт-Петербург}
\end{titlepage}

\section*{Введение}
\vspace{0.8em}

В данной лабораторной работе будем исследовать поведение дискретных динамических систем, задаваемых точечным отображением
\[\vec{x}_{n+1} = \vec{f}(\vec{x}_n),\]
где $\vec{x}_n$ — состояние системы в момент $n$, а $\vec{f}$ —
отображение, определяющее переход к следующему состоянию, на примере логистического отображения и его модификаций.
\vspace{0.8em}

Будем использовать понятия предела, монотонности и ограниченности, изученные в рамках курса математического анализа, в данной работе для исследования рассматриваемых в качестве примера динамических систем последовательностей.
\vspace{0.8em}

\textbf{Смысл работы} состоит в закреплении полученных знаниях на практике. Наша задача - научиться исследовать поведение посследовательностей и изучить свойства последовательностей, задаваемых рекуррентно.

\vspace{0.8em}
\textbf{Ход работы:}
\begin{itemize}
    \item Проанализируем динамику логистического отображения
    при различных значениях параметров и начального условия.
    
    \item Исследуем влияние параметра $r$ на поведение точечных отображений и изменение их качественных свойств.
    
    \item Ознакомимся с понятием неподвижной точки, определим условия её существования и количество.
    
    \item Изучим последовательности, порождаемые отображением, на предмет монотонности и наличия предела.
    
    \item Проанализируем существование циклов порядка $m$ и их свойства для рассматриваемых отображений.
    
    \item Исследуем поведение циклов разных порядков с использованием построения лестниц Ламерея.
    
    \item Подтвердим полученные аналитические результаты с помощью графической визуализации.
\end{itemize}
\clearpage


\section*{Easy}
Прежде чем приступить к практике, обозначим вводимые в рамках работы определения и поясним их.

\textbf{Определение:}
\textit{Логистическим отображением} называется функция вида
\[x_{n+1} = r x_n (1 - x_n),\]
где $r$ — параметр скорости прироста, $x_n$ - состояние системы в некий момент $n$, а $x_{n+1}$ - её же состояние в следующий момент. Также известны свойства логистического отображения, такие как:
\[r \in [0,4],\]
\[\forall n \in \mathbb{N}_0 \quad x_n \in [0,1].\]

\small{\textbf{\textit{NB}:} Логистическое отображение широко используется при описании динамических систем, например при исследовании скорости изменения численности популяций. В таком случае параметр $r$ характеризует скорость роста популяции от момента $n$ до момента $n+1$ (фактор "рождаемости минус смертности")}

\subsection*{Задача 1}
\textbf{Доказать утверждение:}
\[\forall n \in \mathbb{N}, \; \forall r \in (0,1]:
\quad
0 < x_0 < 1 \; \Longrightarrow \; 0 < x_n < 1.\]

\begin{proof}
(по методу математической индукции) \\

\textbf{\small{1) База индукции:}}

По определению логистического отображения $x_n = r x_{n-1} (1 - x_{n-1}).$ 

Тогда при $n=1$ имеем:
\[x_1 = r x_0(1-x_0).\]

Так как $0<x_0<1$, имеем: \[-1<-x_0<0 \iff 0<1-x_0<1\]

Тогда $0<x_0(1-x_0)<1.$ При $r \in (0,1]$ получаем:\[0 < r x_0(1-x_0) < 1.\]

Итак,\[0<x_1<1.\]

\textbf{\small{2) Индукционное предположение:}}

Пусть для некоторого $k = n$ верно:
\[0<x_k<1.\]

Тогда докажем, что и $0<x_{k+1}<1$ через определение логистического отображения:
\[x_{k+1} = r x_k (1 - x_k).\]

Так как $0 < x_k < 1$, то $0 < 1 - x_k < 1.$ Следовательно, $0 < x_k(1 - x_k) < 1.$

При $r \in (0,1]$, получаем
\[0 < r x_k(1 - x_k) < 1.\]

Следовательно,
\[0 < x_{k+1} < 1.\]

\textbf{\small{3) Интерпретация:}}

Значит при $x_0 \in (0,1)$ верно:
\[\forall n \in \mathbb{N}, \; \forall r \in (0,1]:
\quad 0 < x_n < 1.\]

Действительно, если про начальное состояние $x_0$ нам известно, что $x_0 \in (0,1)$, тогда, исходя из формулы логистического отображения и ограничений, наложенных на $r$, после $n$ итераций мы получим $x_n$, также не выходящее за пределы $(0,1)$. Т.е. справедливо будет сказать:
\[\forall n \in \mathbb{N}, \; \forall r \in (0,1]:
\quad
0 < x_0 < 1 \; \Longrightarrow \; 0 < x_n < 1.\]
\end{proof}

\subsection*{Задача 2}
\textbf{1. Как параметр $r$ влияет на поведение функции зависимости $x_{n+1}$ от $x_n$?}

Заметим, что график зависимости $x_{n+1}$ от $x_n$ — это график функции
\[y=f(x)=rx(1-x).\]

Рассмотрим этот график, чтобы изучить влияние параметра $r$ на поведение функции:

\begin{itemize}
    \item $f(0)=0$ и $f(1)=0$, т.е. график проходит через точки $(0,0)$ и $(1,0)$ при любом $r$.
    \item $f(x)=rx(1-x)=-rx^2+rx$ - квадратичная функция, график является параболой с ветвями, направленными вниз.
    \item Вершина параболы достигается при $x=\frac12$:
    \[f\!\left(\frac12\right)=r\cdot\frac12\left(1-\frac12\right)=\frac r4.\]
    Следовательно, параметр $r$ масштабирует график по вертикали: чем больше $r$, тем выше вершина и тем больше значения $x_{n+1}$ при фиксированном $x_n\in(0,1)$.

    \item Так как $\max f(x)=\frac r4$, то при $r \in [0, 4]$ имеем $f(x) \leq 1$ на $[0,1]$, что согласуется с оценкой $x_n \in [0,1].$ При $r>4$ член $x_n$ может оказаться вне этого диапазона.
    \item \textbf{Вывод:} Параметр $r$ не меняет значения $f$ на $x=0, \quad x=1$, но изменяет высоту и крутизну графика: при росте $r$ парабола растягивается вверх, а максимум $\max f(x)=\frac r4$ растёт линейно.
\end{itemize}

\textbf{2. Постройте эту функцию для нескольких различных значений r.}
\begin{figure}[h!]
\centering
\includegraphics[width=0.9\textwidth]{images/logistic_map.png}
\caption{Графики функции $x_{n+1} = r x_{n}(1-x_{n})$ при различных $r$.}
\end{figure}

\subsection*{Задача 3}
\textbf{1. Постройте графики зависимости $x_{n+1}$ от $x_n$ для различных значений $r$.}

По условию варианта ($N=0$) будем рассматривать точечное отображение
\[x_{n+1} = r x_n(1-x_n)(2+x_n), \qquad r \in \left[0;\; \frac{27}{2(7\sqrt{7}-10)}\right].\]

Рассмотрим график этой функции:
\begin{itemize}
    \item $g(0)=0$ и $ g(1)=0.$ Следовательно, график проходит через точки $(0,0)$ и $(1,0)$. Также существует нуль при $x=-2$, однако он не принадлежит отрезку $[0,1]$.
    \item $g(x)=r(2x-x^2-x^3)$ - кубическая функция, графиком является кубическая парабола. 
    \item Рассмотрим $g(x)$ при $x\in[0,1]$: \[x(1-x)(2+x) \ge 0.\]
    так как $\forall x \in [0,1] \quad g(x) \in [0, 1]$, получаем оценку $r \ge 0$
    \item Чтобы найти максимум функции, рассмотрим её производную: \[g'(x)=r(2-2x-3x^2).\]

Из уравнения $g'(x)=0$ получаем: \[2-2x-3x^2=0 \quad \Rightarrow \quad x=\frac{\sqrt7-1}{3}.\]

Эта точка принадлежит $(0,1)$ (т.к. $\frac{\sqrt7-1}{3} \approx 0.55)$ Данная точка экстремума единственна на рассматриваемом промежутке, и проходя через неё $g'(x)$ меняет свой знак с $+$ на $-$, следовательно, $x = \frac{\sqrt7-1}{3}$ - точка максимума функции на $(0,1)$.

    \item Подставим это значение, чтобы оценить $r$:
\[\max g(x)=r(2\frac{\sqrt7-1}{3}-\frac{(\sqrt7-1)^2}{9}-\frac{(\sqrt7-1)^3}{9})=\] 
\[= r(2\frac{\sqrt7-1}{3} - \frac{8-2\sqrt7}{9} - \frac{10\sqrt7-22}{27}) = r\frac{18(\sqrt7-1)-3(8-2\sqrt7)-(10\sqrt7-22)}{27} =\] \[= r\frac{2(7\sqrt7-10)}{27}.\]

Из $\max g(x)\le 1$ следует: \[r \le \frac{27}{2(7\sqrt7-10)}; \qquad r\in [0, \frac{27}{2(7\sqrt7-10)}]\]
\end{itemize}

\begin{figure}[h!]
\centering
\includegraphics[width=1.0\textwidth]{images/N0_map.png}
\caption{Графики функции $x_{n+1} = r x_{n}(1-x_{n})(2+x_{n})$ при различных $r$.}
\end{figure}
\vspace{0.8em}

\textbf{2.  Сделайте вывод о сходстве или различии поведения логистического отображения и точечного отображения из вашего варианта. Чем могут быть вызваны сходства/различия?} \\

Исходя из рассмотренных свойств функции $g(x) = rx(1-x)(2+x)$, нетрудно выявить её сходства и различия сравнительно с функцией логистического отображения $f(x) = rx(1-x)$:

\vspace*{0.5cm}
\textbf{Сходства.}
\begin{itemize}
    \item В обоих случаях имеется точечное отображение вида $x_{n+1} = F(x_{n})$, зависящее от параметра $r$, причём увеличение $r$ приводит к увеличению значений $F(x)$ при фиксированном $x\in(0,1)$ ($r$ вертикально масштабирует график).
    \item Оба отображения имеют нули на $[0,1]$ в точках $x=0$ и $x=1$: \[f(0)=f(1)=0, \qquad g(0)=g(1)=0,\] поэтому их графики проходят через точки $(0,0)$ и $(1,0)$, вне зависимости от $r$.
\end{itemize}

\textbf{Различия.}
\begin{itemize}
    \item Логистическое отображение задаёт квадратичную функцию, а отображение $g(x)=r x(1-x)(2+x)$ - кубическую. Поэтому форма графиков будет отличаться: хотя на [0,1] внешне графики и схожи, кривая $g(x)$ не является параболой $f(x)$ и имеет иное распределение значений по $x$ (даже при фиксированном $r$).
\end{itemize}

Очевидно, что сходства обусловлены общей структурой: оба отображения задаются итеративно и зависят от параметра $r$ как от масштабирующего множителя.

В свою очередь, различия вызваны разной старшей степенью отображений, что меняет форму графика $x_{n+1}$ от $x_{n}$ и, как следствие, характер итераций.
\clearpage


\section*{Normal}
Прежде чем приступать к дальнейшим заданиям, рассмотрим вводимое в рамках работы определение неподвижной точки.

\textbf{Определение.}
Точка $x^*$ называется \textit{неподвижной точкой} отображения $f(x)$, если \[x^* = f(x^*).\]
Иными словами, при применении отображения неподвижная точка переходит сама в себя.

\subsection*{Задача 4}
\textbf{1. Найти все неподвижные точки логистического отображения.}

Рассматривая $x_n = x^*$ в логистическом отображении $x_{n+1} = r x_n (1 - x_n)$, справедливо будет заметить, что:
\[x_{n} = r x_n (1 - x_n).\]

Найдём неподвижные точки, удовлетворяющие этому равенству, решив его:
\[x = r x (1 - x) \Longleftrightarrow r x - r x^2 - x = 0 \Longleftrightarrow\] 
\[ \Longleftrightarrow - r x^2 + (r - 1)x = 0 \Longleftrightarrow\ x \big( - r x + r - 1 \big) = 0\]

Отсюда получаем
\[x = 0,\]
\[- r x + r - 1 = 0.\]

Из второго уравнения:
\[r x = r - 1 \Longleftrightarrow\ x = \frac{r - 1}{r}\]

Итак, мы нашли неподвижные точки:
\[x_1 = 0, \qquad x_2 = \frac{r - 1}{r}.\]

\textbf{2. При каких $r$ отображение имеет одну неподвижную точку? Несколько?}

При $r = 0$, отображение $f(x) = r x (1 - x) = 0$, и уравнение $x = f(x)$ даёт единственное решение: \[x = 0.\]

При $r = 1$, согласно найдённой неподвижной точке, имеем: \[x_2 = \frac{1 - 1}{1} = 0,\]
т.e. обе формулы дают одну и ту же точку. Следовательно, при $r = 1$ неподвижная точка одна. Итак:
\[\begin{cases}
r = 0 \cup r = 1 & \text{одна неподвижная точка},\\
r \in (0,1) & \text{две различные неподвижные точки}.
\end{cases}\]
\vspace{0.5 cm}

\textbf{3. Какое максимальное количество неподвижных точек может иметь логистическое отображение? Почему?}

Уравнение неподвижных точек приводится к квадратному уравнению относительно $x$:
\[x = r x (1 - x) \Longleftrightarrow - r x^2 + (r - 1)x = 0.\]

Квадратное уравнение может иметь не более двух действительных корней. Следовательно, максимальное количество неподвижных точек логистического отображения равно двум.
\subsection*{Задача 5}
\textbf{Условие.}
Пусть $x_0 \in (0;1)$, $r \in (0;1]$, и последовательность $\{x_n\}$ задана логистическим отображением: \[\forall n \in \mathbb{N}: \quad x_n = r x_{n-1} (1 - x_{n-1}),\].

\textbf{1. Доказать, что $\{x_n\}$ монотонно убывает}

\begin{proof}
По \textbf{задаче 1} нам уже известно, что: \[\forall n \in \mathbb{N}, \; \forall r \in (0,1]: \quad 0 < x_0 < 1 \; \Longrightarrow \; 0 < x_n < 1.\]
Таким образом имеем:
\[\begin{cases}
x_n \in (0,1) \\
(1-x_n) \in (0,1) \\
r \in (0,1]
\end{cases}\]
Отсюда следует, что $0 < r(1-x_n) < 1$, тогда при $0 < x_n < 1$ будет верно:
\[x_n \cdot r(1-x_n) < x_n\]
\[x_{n+1} < x_n.\]
Значит, последовательность $\{x_n\}$ строго монотонно убывает.
\end{proof}

\textbf{2. Определить, будет ли последовательность $\{x_n\}$ сходящейся при $r \in (0,1]$. Подтвердить это графически.}

\begin{proof}
По \textbf{пункту 1} последовательность $\{x_n\}$ строго монотонно убывает. По \textbf{задаче 1} также известно, что $\forall n \in \mathbb{N}: \; 0 < x_n < 1$, значит $\{x_n\}$ ограничена снизу нулём. 

Следовательно, по теореме о монотонной ограниченной последовательности у $\{x_n\}$ существует конечный предел.
\end{proof}

\begin{figure}[h]
    \centering
    \includegraphics[width=0.9\textwidth]{images/seq.png}
    \caption{График последовательности $x_n$ при $r \in (0;1]$.}
\end{figure}

\small{\textbf{\textit{NB}:} По рис.3 видно, что $\{x_n\}$ сходится к нулю. Нетрудно доказать это аналитически: $\lim\limits_{n \to \infty} x_{n+1} = \lim\limits_{n \to \infty} r x_n (1-x_n).$ Так как последовательность задана рекуррентно, то $\lim\limits_{n \to \infty} x_{n+1} = \lim\limits_{n \to \infty} x_n$, поэтому
$\lim\limits_{n \to \infty} x_n = r\lim\limits_{n \to \infty} x_n (1-\lim\limits_{n \to \infty} x_n ).$

После преобразований получили:
\[\lim\limits_{n \to \infty} x_n=0 \quad \text{или} \quad r(1-\lim\limits_{n \to \infty}x_n)=1.\]

При $r \in (0;1]$ имеем $\frac{r-1}{r} \le 0$, а так как $x_n > 0$ для всех $n$, то предел не может быть отрицательным. Следовательно,
\[\forall x_n \in \{x_n\} \quad\lim\limits_{n \to \infty} x_n = 0.\]}


\subsection*{Задача 6}
\textbf{Условие.}
Пусть $r \in (2;3)$, $x_{2n} > x^*$, $x_{2n+1} < x^*$, где $x^* = \dfrac{r-1}{r}$ — ненулевая неподвижная точка логистического отображения.

\textbf{1. Исследовать монотонность подпоследовательностей $\{x_{2n}\}$ и $\{x_{2n+1}\}$.}

\textbf{1)} Так как нам известны оценки $x_{2n} > x^*$, $x_{2n+1} < x^*$, сравним $x_n = f(x_{n-1})$ и $x^* = f(x^*)$:
\[f(x)-f(x^*) = rx(1-x)-\dfrac{r-1}{r} = rx - rx^2 - \dfrac{r-1}{r} = rx - rx^2 - \dfrac{r-1}{r} +x - x =\] \[  = \Bigl(-\dfrac{r-1}{r} + rx - x\Bigr) - rx^2 + x = (rx - 1) \cdot \dfrac{r-1}{r} + (-rx + 1) \cdot x =\] \[ = -\Bigl(x-\dfrac{r-1}{r}\Bigr)\Bigl(rx-1 \Bigr) = -r\Bigl(x-\dfrac{r-1}{r}\Bigr)\Bigl(x-\frac{1}{r}\Bigr).\]

При $r \in (2;3)$ имеем: $0<\dfrac{1}{r}<x^*<1$.

Из полученного следует:
\begin{itemize}
    \item Если $x>x^*$, то $(x-x^*)>0$ и $\Bigl(x-\dfrac{1}{r}\Bigr)>0$, значит $f(x)-f(x^*)<0,$ то есть\[x>x^* \Rightarrow f(x)<f(x^*)=x^*.\]
    \item Если $\frac{1}{r}<x<x^*$, то $(x-x^*)<0$ и $\Bigl(x-\dfrac{1}{r}\Bigr)>0$, значит $f(x)-f(x^*)>0,$ то есть
\[x<x^* \Rightarrow f(x)>f(x^*)=x^*.\]
\end{itemize}

Таким образом, отображение $f$ «перебрасывает» точку через $x^*$: если $x_n$ лежит выше $x^*$, то $x_{n+1}=f(x_n)$ оказывается ниже, и наоборот.

\textbf{2)} Так как одна итерация меняет сторону относительно $x^*$, удобно рассмотреть двушаговое отображение, определяющее каждую из подпоследовательностей: \[g(x)=f(f(x)).\]
\[x_{2n+2}=g(x_{2n}), \qquad x_{2n+3}=g(x_{2n+1}).\]

Исследуем поведение $g(x)$ с помощью производной:
\[g'(x)=f'(f(x))f'(x)\] 
\[g'(x^*)=(f'(x^*))^2=(2-r)^2.\]

При $r \in (2;3)$ имеем
\[0<(2-r)^2<1,\]

Это значит:
\begin{itemize}
\item $g(x)$ сохраняет сторону $x$ ($g'(x^*)>0$) (если $xRx^*$, то и $g(x)Rx^*$),
\item $g(x)$ уменьшает расстояние до $x^*$ ($|g'(x^*)|<1$). (т.е. $|g(x)-x^*|<|x-x^*|$)
\end{itemize}

\textbf{3)} Так как $g(x)$ уменьшает расстояние до $x^*$, то
\[x_{2n+2}-x^* < x_{2n}-x^*, \qquad x_{2n+2}<x_{2n}\]

Значит, подпоследовательность $\{x_{2n}\}$ монотонно убывает.

Аналогично, для $x_{2n+1}$: \[x_{2n+3}>x_{2n+1}.\]

Следовательно, подпоследовательность $\{x_{2n+1}\}$ монотонно возрастает.

\vspace{0.3 cm}
\textbf{2. Графически проверить это утверждение.}

На рис. 4 видно, что чётные значения располагаются по одну сторону от $x^*$ и монотонно приближаются к $x^*$, а нечётные — по другую сторону и также монотонно приближаются к $x^*$.

\begin{figure}[h]
    \centering
    \includegraphics[width=0.9\textwidth]{images/stilldot.png}
    \caption{Чётная и нечётная подпоследовательности ${x_n}$ относительно $x^*$ при $r \in (2;3)$.}
\end{figure}

\subsection*{Задача 7}
\textbf{1. Для отображения $g(x_n)$, заданного вариантом, найти все неподвижные точки.}
Напомним себе данное нам отображение:
\[x_{n+1} = g(x_n),\qquad g(x) = r x(1-x)(2+x),\qquad r \in \left[0;\ \frac{27}{2(7\sqrt{7}-10)}\right].\]
По определению, $x^\ast$ — неподвижная точка, если $g(x^\ast)=x^\ast$.
Решим уравнение: \[r x^\ast(1-x^\ast)(2+x^\ast) = x^\ast.\]
Очевидно, что $x_1=0$ — неподвижная точка. Для $x^\ast\neq 0$:
\[r(1-x^\ast)(2+x^\ast)=1 \Longleftrightarrow r(2-x^\ast-(x^\ast)^2)=1 \Longleftrightarrow (x^\ast)^2+x^\ast+\left(\frac{1}{r}-2\right)=0.\]
Отсюда получаем единственную ненулевую неподвижную точку: \[x_{2}=\frac{-1+\sqrt{9-\frac{4}{r}}}{2}.\]
\small{\textbf{\textit{Note}:} Значение будет вещественно тогда и только тогда, когда \[9-\frac{4}{r}\ge 0\Longleftrightarrow r\ge \frac{4}{9}.\]}
\textbf{2. Найти диапазон $r$, при котором $\{x_n\}$ монотонно сходится к нулю.} \\
Найдём условия, при которых для данной $\{x_n\}$ выполняется:
\[ \forall n: 0 < x_{n+1} < x_n\] \[\lim_{n\to\infty}x_n=0.\]
Для $x\in(0,1)$ рассмотрим формулу членов последовательности:
\[g(x) = rx(1-x)(2+x) \Longleftrightarrow \frac{g(x)}{x}=r(1-x)(2+x)\]
На $(0,1)$ имеем $0 < (1-x)(2+x) < 2$.
Если $0<r\le \frac12$, то для любого $x\in(0,1)$ получаем
\[0<r (1-x)(2+x) < r\cdot 2 \le 1 \Longrightarrow 0<g(x)<x.\]
Применяя это к $x=x_n$, заключаем:
\[\forall n: 0<x_{n+1}<x_n,\]
то есть $\{x_n\}$ ограничена снизу и строго убывает, следовательно у $\{x_n\}$ существует предел.

\textbf{2.* Докажем, что предел этой функции равен неподвижной точке.} \\
Так как $x_n \to \lim_{n\to\infty}x_n\ $, то и $x_{n+1} \to \lim_{n\to\infty}x_n.$

Функция $g(x)=r x(1-x)(2+x)$ непрерывна на $\mathbb{R}$.
Из её непрерывности следует: \[\lim_{n\to\infty} g(x_n) =  g\!\left(\lim_{n\to\infty} x_n\right)\]

С другой стороны,
\[\lim_{n\to\infty} x_{n+1}= \lim_{n\to\infty} g(x_n)\]

Следовательно,
\[\lim_{n\to\infty}x_n = \lim_{n\to\infty} g(x_{n+1}) = g(\lim_{n\to\infty}x_n),\]
то есть предел последовательности является неподвижной точкой отображения.

\small{\textbf{\textit{NB}:} Аналогичное утверждение было приведено в качестве теоретической выкладки в техзадании, но меня смутило то, что при рассматриваемых $\{x_n\}$ и $r$ приведённое утверждение как будто некорректно применять. Доказательство выше было перестраховкой на всякий.}

Покажем, что при $0<r\le \frac12$ других неотрицательных неподвижных точек, кроме $0$, нет.
Действительно, для $r\le \frac12$ выполнено $9-\frac{4}{r} \le 9-8 = 1,$
поэтому \[\frac{-1+\sqrt{9-\frac{4}{r}}}{2}\le 0\]
Следовательно, при $0<r\le \frac12$ единственная неподвижная точка на $[0,1]$ — это $0$.
Итак, при $x_0\in(0,1)$ последовательность $\{x_n\}$ монотонно сходится к нулю при $0<r\le \frac12.$
При $r=0$ имеем $x_1=0$, то есть сходимость к нулю также имеет место.
Итоговый диапазон: \[r \in \left[0;\ \frac12\right].\]

    \textbf{3. Построить графики зависимости $x_n$ от $n$ для различных значений r.}

\begin{figure}[h]
    \centering
    \includegraphics[width=0.9\textwidth]{images/N0_seq.png}
    \caption{Графики зависимости $x_{n+1}$ от $x_n$ при разных $r$.}
\end{figure}

\clearpage

\section*{Hard}

Прежде чем приступать к дальнейшим заданиям, рассмотрим вводимые в рамках работы определения и понятия.

\textbf{Определение.}  
Точка $x^*$ называется $\textit{периодической точкой порядка $m \in \mathbb{N}$}$, если \[f^{(m)}(x^*) = x^*,\]
где $f^{(m)}$ — $m$-кратная композиция $f$ с собой.

\textbf{Определение.}  
Множество точек \[\{x^*, f(x^*), f^{(2)}(x^*), \dots, f^{(m-1)}(x^*)\}\] называется $\textit{циклом порядка $m$}$, если каждая из этих точек переходит в следующую, а последняя — в первую.

\subsection*{Задача 8.}
Введём величину \[r_\infty \approx 3.5699456.\]
\textbf{1. Определить, как меняется длина цикла порядка $m$ при $r \in (3, r_{\infty}]$}

Всё так же рассматриваем логистическое отображение: \[x_{n+1} = r x_n (1 - x_n)\]

Вспомним его неподвижные точки: \[x^*=0\]\[x^*=\dfrac{r-1}{r}.\]

Нам уже известно, что при $1<r<3$ последовательность $\{x_n\}$ стремится к точке $x^*$.  
При $r=3$ величина $f'(x^*)=2-r$ по модулю становится равной $1$, и дальнейшее увеличение $r$ приводит к тому, что равенство $x_{n+1}=x_n$ уже не описывает предельное поведение последовательности, значит, возможно, на этом промежутке будут циклы большего порядка.

Для исследования возможного цикла порядка $2$ рассмотрим уравнение\[f^{(2)}(x)=x.\]
Вычисления дают
\[f^{(2)}(x)-x=-x\,(rx-r+1)\,\bigl(r^2x^2-r^2x-rx+r+1\bigr).\]
Первые два множителя соответствуют уже найденным неподвижным точкам.  
Оставшиеся решения определяются квадратным уравнением \[r^2x^2-r^2x-rx+r+1=0,\] откуда\[x_{1,2}=\dfrac{r+1\pm\sqrt{(r-3)(r+1)}}{2r}.\]

Из выражения под корнем видно, что вещественные решения появляются при \[(r-3)(r+1)\ge 0,\] то есть при $r\ge 3$. Тогда наше предположение верно: начиная с $r>3$ существуют точки, образующие цикл порядка $m=2$.

При дальнейшем увеличении параметра $r$ аналогичный анализ уравнения \[f^{(4)}(x)=x\]показывает появление новых решений, не совпадающих с решениями предыдущих уравнений, что соответствует возникновению цикла порядка $4$. ЧТобы в этом убедиться, рассмотрим многочлены:
\[P_1(x)=f(x)-x,\qquad P_2(x)=f^{(2)}(x)-x,\qquad P_4(x)=f^{(4)}(x)-x.\]
Заметим, что из равенства $f(x)=x$ следует $f^{(2)}(x)=x$ и $f^{(4)}(x)=x$, а из $f^{(2)}(x)=x$ также следует $f^{(4)}(x)=x$. Поэтому множество корней $P_1$ содержится в множестве корней $P_2$, а множество корней $P_2$ содержится в множестве корней $P_4$.

Так как $f(x)=rx(1-x)$ является многочленом степени $2$, то $f^{(k)}(x)$ является многочленом степени $2^k$. Следовательно,
\[\deg P_1=2,\qquad \deg P_2=4,\qquad \deg P_4=16.\]
Уравнение $P_4(x)=0$ имеет не более $16$ корней (с учётом кратных), причём оно уже содержит все корни уравнения $P_2(x)=0$, которых не более $4$. Значит, в общем случае уравнение $f^{(4)}(x)=x$ обладает дополнительными корнями, отличными от корней уравнения $f^{(2)}(x)=x$. Эти новые корни и соответствуют появлению цикла порядка $m=4$.

Аналогично, при переходе к $f^{(8)}(x)=x$ степень многочлена возрастает до $2^8=256$, и уравнение $f^{(8)}(x)=x$ содержит в качестве части решений как корни $f^{(4)}(x)=x$, так и новые корни, не встречавшиеся ранее; они соответствуют циклам более высокого порядка.

Отсюда следует, что при увеличении параметра $r$ длина цикла не может увеличиваться плавно: период $m$ всегда является натуральным числом, и при изменении $r$ он может изменяться только скачками (переходом от одного целого значения к другому).

Таким образом, при $r \in (3,r_\infty)$ длина цикла $m$ скачкообразно, но монотонно изменяется при росте параметра $r$.

\textbf{2. Экспериментально установить ограничения, накладываемые на $m$ для цикла}

Для исследования использовалась функция \textbf{estimate\_period}, которая по хвосту последовательности $\{x_n\}$ определяет минимальное $m$, удовлетворяющее условию
\[|x_{n+m} - x_n| < \varepsilon\] для всех $n$ на рассматриваемом участке.

Вычисления проводились для $r \in (3, r_\infty)$. Для каждого $r$ строилась последовательность $\{x_n\}$ и оценивалась длина цикла $m$. \\
\clearpage

\begin{lstlisting}
import numpy as np

def estimate_period(seq: np.ndarray, skip: int = 500, tail_len: int = 1000, max_m: int = 512, eps: float = 1e-6) -> int:
    n = len(seq)
    end = min(n, skip + tail_len)
    tail = seq[skip:end]

    for m in range(1, max_m + 1):
        a = tail[m:]
        b = tail[:-m]
        if np.all(np.abs(a - b) < eps):
            return m
    return 0

def scan_periods(r_values: np.ndarray,x_0: float,n_steps: np.int64) -> np.ndarray:
    periods = np.empty(len(r_values), dtype=int)

    for i, r in enumerate(r_values):
        seq = generate_sequence(x_0, float(r), n_steps, "log")
        periods[i] = estimate_period(seq=seq)
    return periods
\end{lstlisting}

По полученному графику (рис. 6) зависимости $m(r)$ установлено следующее:
\begin{itemize}
    \item возможные значения периода имеют вид $m = 0, 2, 4, 8, 16, \dots$;
    \item увеличение $m$ происходит скачкообразно при увеличении $r$;
    \item других значений на интервале $(3, r_\infty]$ не наблюдается.
\end{itemize}

При допущении, что цикла периода 0 тривиальным образом не существует, экспериментально установлено, что на интервале $(3, r_\infty]$ длина цикла принимает только значения, являющиеся степенями двойки: \[m = 2^k, \quad \forall k \in \mathbb{N}.\]

\begin{figure}[h]
    \centering
    \includegraphics[width=0.8\textwidth]{images/experimental.png}
    \caption{Зависимость периода цикла от параметра $r$ для логистического отображения}
\end{figure}
\clearpage

\subsection*{Задача 9.}
\textbf{Определение.}
\textit{Лестница Ламерея} - это графический способ исследования динамики отображения $f(x)$, осуществляемый следующим образом:
\begin{itemize}
    \item строится график функции $y = f(x)$ и прямая $y = x$;
    \item из точки $(x_0, 0)$ проводится вертикаль до пересечения с графиком $y=f(x)$;
    \item далее проводится горизонталь до прямой $y=x$;
    \item процесс повторяется.
\end{itemize}

\textbf{1. Реализовать функцию, которая строит лестницу Ламерея для заданного $r$}.
\begin{lstlisting}
import numpy as np

def logistic_map(x: np.ndarray[np.float64], r: float) -> np.ndarray[np.float64]:
    return r * x * (1 - x)

def build_staircase(x_0: float, r: float, n_steps: np.ndarray) -> tuple[np.ndarray, np.ndarray]:

    x_array = np.empty(2 * int(n_steps) + 1)
    y_array = np.empty(2 * int(n_steps) + 1)

    x_n = x_0
    x_array[0] = x_n
    y_array[0] = x_n

    for k in range(int(n_steps)):
        x_next = float(logistic_map(x_n, r))

        x_array[2 * k + 1] = x_n
        y_array[2 * k + 1] = x_next

        x_array[2 * k + 2] = x_next
        y_array[2 * k + 2] = x_next

        x_n = x_next

    return x_array, y_array
\end{lstlisting}

В функции \textbf{build\_staircase} на каждом шаге вычисляется $x_{n+1} = r x_n (1 - x_n),$ после чего добавляются две точки ломаной: $(x_n, x_{n+1}) \quad \text{и} \quad (x_{n+1}, x_{n+1}).$ Таким образом реализуется стандартная схема лестницы Ламерея.

\begin{figure}[h]
    \centering
    \includegraphics[width=0.7\textwidth]{images/stairs_1.png}
    \caption{Лестница Ламерея для логистического отображения (пример 1)}
\end{figure}

\begin{figure}[h]
    \centering
    \includegraphics[width=0.7\textwidth]{images/stairs_2.png}
    \caption{Лестница Ламерея для логистического отображения (пример 2)}
\end{figure}

\begin{figure}[h]
    \centering
    \includegraphics[width=0.7\textwidth]{images/stairs_3.png}
    \caption{Лестница Ламерея для логистического отображения (пример 3)}
\end{figure}

\textbf{2. Сделать по графикам вывод о циклах различных порядков}.

Из графиков видно, что после нескольких итераций лестница Ламерея начинает периодически описывать некоторую замкнутую фигуру.

На рис.9 при $r=3.2$ лестница попеременно проходит через две различные точки диагонали $f(x)=x$, то есть через две точки вида $(x_n,x_n)$. Это означает выполнение $x_{n+2}=x_n$ при $x_{n+1}\ne x_n$, что описывает цикл порядка $m=2$.

На рис.10 при $r=3.5$ число различных диагональных точек увеличивается: траектория проходит через четыре различные точки $(x_n,x_n)$ до возвращения в исходную. Тем самым выполняется $x_{n+4}=x_n$, что соответствует циклу порядка $m=4$.

При дальнейшем увеличении параметра $r$ (рис.11, $r=3.5680$) число различных диагональных точек возрастает ещё больше, что свидетельствует о возникновении циклов более высокого порядка.

Таким образом, порядок цикла определяется числом различных точек пересечения лестницы с диагональю $y=x$: при двух таких точках реализуется цикл порядка $2$, при четырёх — порядка $4$, и при увеличении $r$ наблюдается последовательное возрастание порядка цикла.
\clearpage

\subsection*{Задача 10.}

\textbf{1. Экспериментально исследовать изменение длины цикла при изменении параметра $r$ для заданного вариантом отображения}.

\begin{figure}[h]
    \centering
    %\includegraphics[width=0.8\textwidth]{src/images/hard_N0_period.png}
    \caption{Зависимость периода цикла от параметра $r$ для отображения}
\end{figure}

\begin{figure}[h]
    \centering
    %\includegraphics[width=0.7\textwidth]{src/images/hard_N0_cobweb_r1.png}
    \caption{Лестница Ламерея для отображения(пример 1)}
\end{figure}

\begin{figure}[h]
    \centering
    %\includegraphics[width=0.7\textwidth]{src/images/hard_N0_cobweb_r2.png}
    \caption{Лестница Ламерея для отображения (пример 2)}
\end{figure}

\textbf{2. Сравнить полученные графики с графиками логистического отображения}.

\end{document}
